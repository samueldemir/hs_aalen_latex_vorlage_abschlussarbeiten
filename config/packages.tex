\usepackage[utf8]{inputenc}
\usepackage[T1]{fontenc}
\usepackage[ngerman]{babel}

% Blindtext to test
\usepackage{blindtext}

% Betetr font
\usepackage{lmodern}

\usepackage[printonlyused]{acronym}

% Load mathematic environment american mathematics society
\usepackage{amsmath}

% Load more mathematic symbols
\usepackage{amssymb}

% Import setspace to make distances between paragraphs
\usepackage{setspace}

% Importing csquotes because babel and biblatex use it
\usepackage[german=guillemets]{csquotes}

% Color settings
\usepackage{xcolor}
\xdefinecolor{sourcecodebackground}{RGB}{240, 240, 240}
\usepackage{tcolorbox}
\tcbset{width=\textwidth, boxrule=0pt, colback=orange, arc=0pt, auto outer arc, left=0pt, right=0pt}

% Float Package
\usepackage{float}

% Package for graphics
\usepackage{graphicx}

% Package for enumerates and for vspace between each item
\usepackage{enumitem}

% Header and Footer adjustments
\usepackage[autooneside=false]{scrlayer-scrpage}
\pagestyle{scrheadings}
\automark[section]{chapter}
\automark*[section]{}
\rohead{\rightmark}
\lohead{\leftmark}
\chead{}

% Falls im Header die Kapitel/Abschnittsnummern nicht angezeigt werden sollen
%\renewcommand*{\chaptermarkformat}{}
%\renewcommand*{\sectionmarkformat}{}
%\renewcommand*{\subsectionmarkformat}{}

% KOMA Options
\KOMAoptions{
	fontsize=12pt,
	twoside=false,
	DIV=8,
	titlepage=true,
	parskip=full,
	listof=totoc,
	index=totoc,
	bibliography=totoc,
	bibliography=openstyle,
	headings=big,
	headsepline=true,
	listof=entryprefix,
	numbers=noenddot
}

% Inhaltsverzeichnis
\DeclareTOCStyleEntries[pagenumberwidth=20pt]{tocline}{chapter,section,subsection}
% If table of contents should be adjusted in vertical space
%\DeclareTOCStyleEntry[beforeskip=.5cm]{chapter}{chapter}
%\DeclareTOCStyleEntry[beforeskip=.2cm]{section}{section}
%\DeclareTOCStyleEntry[beforeskip=.5cm]{default}{subsection}

\renewcommand{\listoflofentryname}{Abb.}
\renewcommand{\listoflotentryname}{Tab.}

% Change Fonts of Header and chapter and chapterentrys in toc -> see KOMA Guide
\addtokomafont{disposition}{\rmfamily\bfseries}
\addtokomafont{pagehead}{\rmfamily\upshape\small}

% #######################################################################################
% ##############################SOURCE CODE ADJUSTMENTS##################################
% #######################################################################################
% Creates custom source code style
\usepackage{listings}
% Used for appearence QC. in listings
% It needs that koma has to understand lol listoflist format
\usepackage{scrhack}			 
% Renaming chapter name for content
\renewcommand{\lstlistlistingname}{Quellcodeverzeichnis}
% Renaming listingsnames
\renewcommand{\lstlistingname}{Quellcode}
% Add new command that koma understand what to do and that it can read 'lol' generated file
\newcommand{\listoflolentryname}{Quellc.}
\AfterTOCHead[lol]{\renewcommand*\autodot{:}}% NEU

\lstdefinestyle{configs}{%
	basicstyle=\small\ttfamily,
	showstringspaces=false,
	keywordstyle=\bfseries,
	stringstyle=\sffamily,
	numbers=left,
	breaklines=true,
	numberstyle=\small,
	numbersep=10pt,
	stepnumber=1,
	upquote=true,
	aboveskip=10pt,
	belowskip=5pt,
	captionpos=b,
	backgroundcolor=\color{sourcecodebackground},
	rulesep=0pt,
	frame=single,
	framesep=0pt,
	framerule=1pt,
	xleftmargin=25pt,
	framexleftmargin=24pt,
	framexrightmargin=-1pt,
	framextopmargin=5pt,
	framexbottommargin=5pt,
	gobble=2,
	float=htbp,
	literate={Ö}{{\"O}}1{Ä}{{\"A}}1{Ü}{{\"U}}1{ß}{{\ss}}1{ü}{{\"u}}1{ä}{{\"a}}1{ö}{{\"o}}1{¡\\)}{{\textbackslash)}}1,
}
\usepackage{textcomp}

\lstdefinestyle{pythonlisting}{%
	language=Python,
	style=configs
}

\lstdefinestyle{texlisting}{%
	language=[LaTeX]TeX,
	style=configs
}

% #######################################################################################
% #######################################################################################
% #######################################################################################

% Creates custom index table
% #######################################################################################
\usepackage[original]{imakeidx}
\makeindex[title=Stichwortverzeichnis, options=-s mystyle]
%\usepackage{filecontents} % wurde von der internen environment "filecontents" abgelöst -> nciht mehr notwendig zu laden
\begingroup\newif\ifmy
\IfFileExists{config/mystyle.ist}{}{\mytrue}
\ifmy
\begin{filecontents}[overwrite]{config/mystyle.ist}
	headings_flag  1 % wir benutzen Überschriften
	heading_prefix "{\\bfseries " % und setzen sie fett
	heading_suffix "\\hfil}\\nopagebreak\n"% und links, nach ihnen kein Seitenumbruch
	delim_0 "\\dotfill" % Punktzeile zwischen Einträgen und Seitenzahlen (Ebene 0)
	delim_1 "\\dotfill" % Punktzeile zwischen Einträgen und Seitenzahlen (Ebene 1)
	delim_2 "\\dotfill" % Punktzeile zwischen Einträgen und Seitenzahlen (Ebene 2)
	delim_r "--" % Trenner zwischen Start und Ende eines Seitenbereiches
	suffix_2p "\\,f." % Suffix bei einem bereich aus 2 Seiten
	suffix_3p "\\,ff." % Suffix bei einem bereich aus 3 Seiten
\end{filecontents}
\fi\endgroup
% #######################################################################################

% Import bibliography
\usepackage[style=alphabetic, isbn=false, backend=bibtex]{biblatex}
\addbibresource{bibliography/literatur.bib}
%\bibliographystyle{unsrt}

% Damit bei Warnungen umgebrochen wird, wenn es zu hbox Meldungen kommt.
\emergencystretch=12pt

% For links (must be loaded at the end of all packages)
\usepackage[pdfauthor={\VarAuthor}, pdftitle={\VarTitel}, pdfsubject={\VarFachgebiet}, pdfkeywords={\VarKeywords}, hidelinks, breaklinks=true, pdffitwindow=true, pdfpagelayout=SinglePage]{hyperref}