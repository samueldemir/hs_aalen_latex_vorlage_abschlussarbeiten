\pdfbookmark{Vorwort}{chap:vorwort}
\chapter*{Vorwort}
	\label{chap:vorwort}
	Das Vorwort befindet sich vor dem eigentlichen Text der Arbeit. Es hat mit diesem keinen systematischen Zusammenhang, sondern gibt dem Autor Gelegenheit, loszuwerden, was er über seine Arbeit und ihren Entstehungsprozeß auf dem Herzen hat. Hier erscheinen daher Informationen über die Ursprünge der Arbeit, Motive für die Themenwahl, besondere Schwierigkeiten, aber insbesondere auch Unterstützung, die der Autor während der Arbeitszeit erfahren hat. Am Schluß findet sich deswegen normalerweise eine Menge von Danksagungen. Handelt es sich um eine umfangreichere Arbeit, kann man ohne weiteres pro Person angeben, in welcher Weise sie einem geholfen hat.

Das Vorwort ist klar von der Einleitung unterschieden. Man kann im Vorwort zwar diejenigen Ziele formulieren, welche außerhalb der Sache bzw. der Disziplin liegen. Das betrifft insbesondere die angezielte Leserschaft und die Frage, was sie mit dem Werk machen soll. Der Inhalt des Werkes dagegen kommt im Vorwort nicht in systematischer Weise vor. Aus demselben Grunde kann es auch keine Querverweise zwischen dem Vorwort und der eigentlichen Arbeit geben.

Wie im Abschnitt über Objektivität erläutert wird, bleibt der Autor eines Werks in dessen Text im Hintergrund. Da das Vorwort nicht Bestandteil des eigentlichen Textes ist, bildet es auch hierzu eine Ausnahme. Es ist im Vorwort ohne weiteres üblich, von sich selbst zu sprechen.